\documentclass{article}
\usepackage{parskip}
\usepackage{amsmath}
\usepackage{float}
\usepackage{listings}
\newfloat{lstfloat}{H}{lop}
\floatname{lstfloat}{Listing}
\def\lstfloatautorefname{Listing}
\usepackage{pgfplots}
\usepackage[letterpaper, portrait, margin=1in]{geometry}
\usepackage{booktabs}

\setlength{\belowcaptionskip}{10pt}

\title{IE4163 Assignment 1: Factorial Experimental Design}
\author{John Bradley}
\date{\today}

\begin{document}
  \maketitle

  \section{Experiment Design}

  Atlanta's Hartsfield-Jackson airport is the busiest airport in the world by
  passenger volume and serves as Delta's main connecting hub. For this
  experiment, the relationship between three factors will be compared against
  the percentage of on-time flights: flight operation, airline, and season.

  Flight operation, meaning an arrival or depature flight, are relatively even
  as there is a limited amount of space at the airport, and for every plane
  that arrives, typically another plane is leaving. This can compare possible
  issues that originate at the aiport or from an airport elsewhere.

  Two airlines will be observed as part of this experiment. Delta is the
  primary operator at this airport, while Endeavor Air is Delta's regional
  operator. This will compare operational efficiencies between the two
  connected airlines and possible differences between larger and smaller 
  planes.

  Finally, sasons will be compared, as weather between these two
  categories of months will be different - hot weather months will see more
  storms while cold weather months will see snow and ice-related delays at
  airports served by Atlanta elsewhere in the north. The specific months chosen
  for each season will be January for Winter, April for Spring, July for
  Summer, and October for Fall.

  On-time flight percentages are based on the cumulative number of flights
  recorded within a month. All data comes from the United States Department of
  Transportation Bureau of Transportation Statistics and will be using data
  from 2023. Percentages are used because average monthly operations between
  the two airlines aren't completely equal, with Delta operating approximately
  15,000 flights monthly and Endeavor Air flies approximately 5,000 monthly.

  \section{Observation}
  
  \begin{table}[H]
    \centering

    \begin{tabular}{ c | c c c c }
      \toprule
                      & On-Time Flights (\%)  & Flight Operation  & Airline   & Season    \\
      \midrule
      Observation 1   & 81.71                 & Arrival           & Delta     & Winter    \\
      \midrule
      Observation 2   & 82.75                 & Arrival           & Delta     & Spring    \\
      \midrule
      Observation 3   & 78.81                 & Arrival           & Delta     & Summer    \\
      \midrule
      Observation 4   & 92.59                 & Arrival           & Delta     & Fall      \\
      \midrule
      Observation 5   & 81.72                 & Arrival           & Endeavor  & Winter    \\
      \midrule
      Observation 6   & 89.78                 & Arrival           & Endeavor  & Spring    \\
      \midrule
      Observation 7   & 86.73                 & Arrival           & Endeavor  & Summer    \\
      \midrule
      Observation 8   & 94.41                 & Arrival           & Endeavor  & Fall      \\
      \midrule
      Observation 9   & 74.04                 & Departure         & Delta     & Winter    \\
      \midrule
      Observation 10  & 76.64                 & Departure         & Delta     & Spring    \\
      \midrule
      Observation 11  & 69.22                 & Departure         & Delta     & Summer    \\
      \midrule
      Observation 12  & 90.14                 & Departure         & Delta     & Fall      \\
      \midrule
      Observation 13  & 83.56                 & Departure         & Endeavor  & Winter    \\
      \midrule
      Observation 14  & 91.43                 & Departure         & Endeavor  & Spring    \\
      \midrule
      Observation 15  & 83.67                 & Departure         & Endeavor  & Summer    \\
      \midrule
      Observation 16  & 94.14                 & Departure         & Endeavor  & Fall      \\
      \bottomrule
    \end{tabular}
    \caption{On-Time Flights}
    \label{tab:exectime}
  \end{table}

  \section{Analyzation and Interaction Effects}

  It stands out that flights originate in the fall, regardless of flight
  operation or airline, tend to be the most likely to be on-time, with the
  highest percentages. This is likely due to more calm and predictable weather
  in the fall, as sprint and summer storms cause the airport to halt flights,
  and unexpected snow-storms and holiday travel may inpact the fall.
  
  Additionally, flights operated by Endeavor Air tend to perform well, with 
  percentages that are 5 - 10\% higher. This may be due to the sheer volume of
  flights that Delta performs, with a higher chance of delays, while Endeavor
  can better focus on the fewer flights that they operate.

  For Delta flights, it seems like flight operation has an effect on if a 
  flight is delayed. Delta flights originating from Atlanta have a tendency to
  be delayed while arrivals are less effected. I believe there is an 
  interaction here because flights by Endeavor have a \textless 5\% affect on 
  on-time flights, while for Delta, departures are \textgreater 5\%, except in the fall.

  \section{Conclusion}

  While there's not much that I personally can do to make improvements to the
  real world results of Delta and Endeavor Air's flight operations, this could
  indicate to Delta that the can make improvements by focusing on possible
  factors that affect on-time performance, specifically processes that cause
  flights that leave Atlanta to be delayed. The data I've collected does not
  look into specific reasons why flights are delayed, but it can narrow it down
  to delays caused by operations at Atlanta.

\end{document}
